\chapter{Estado del Arte}

El concepto de Smart House (casa inteligente) ha sido objeto de investigación por muchos años, en los cuales no se ha logrado estandarizar o llegar a un común acuerdo de su significado, diseño e implementación. Esto, debido a la constante evolución que presenta la tecnología, como se evidencia ante el desarrollo del internet de la cosas (IoT); no obstante, se han propuesto y desarrollado una gran variedad de modelos y prototipos, con el objetivo de abarcar las posibles alternativas para este ambiente. \\

A lo largo del crecimiento de los entornos inteligentes, como Smart House, se han realizado investigaciones con múltiples orientaciones. Las cuales están centradas en factores sociales como la comodidad y la seguridad, sin dejar de lado factores ambientales como el ahorro energético. Como se ha mencionado anteriormente no hay un consenso claro respecto a Smart House, por eso posee estos diferentes enfoques. En cuanto a una parte más técnica, estos procesos inteligentes están compuestos por software, hardware y firmware.\\

Smart House es un término que está muy ligado a la automatización de la casa desde 1980, ya que en este tiempo se veía en la ciencia ficción las casas como “máquinas para vivir”, teniendo diversos enfoques y divagando por diferentes temas a incluir, como telecomunicaciones en la línea eléctrica, también mencionando la capacidad de controlar diferentes sistemas de la casa, entre otras ideas que se generaron desde este tiempo \cite{Gross1998}. Con el avance de la tecnología ha sido posible una visión más tangible de estas primeras ideas que abrieron paso al concepto de Smart House.\\

% Recientemente,

Las investigaciones hacia el entorno de Smart House se enfocan en monitorear y/o controlar múltiples aspectos de una casa. Para realizar esta tarea físicamente se usa un hardware, en el cual se ven inmersos la unidad central de procesamiento, los sensores y los actuadores. La unidad central de procesamiento se encarga del monitoreo y control del entorno, por lo cual, se usan diferentes sistemas embebidos para realizar esta tarea. Entre algunos de estos sistemas, se encuentra la familia de microcontroladores PIC.\\

Estos microcontroladores PIC se usan para implementar módulos de emisión y recepción de un control remoto, el cual posee comunicación por medio de radiofrecuencia (RF) e infrarrojo (IR), de esta manera permite la interacción en toda la casa, como hizo Hamed en \cite{Hamed2012}. Además, el sistema basado en PIC, logra interactuar con el entorno por medio del servicio de mensajes cortos (SMS), usando un módulo del sistema global para las comunicaciones móviles (GSM) que se comunica con el microcontrolador, tal como desarrollo Datta \cite{Datta2014}.\\

Adicional a esto, autores como Behan \cite{Behan2013}  y Cheuque \cite{Cheuque2015} han usado mini computadoras o computadoras de placa simple (SBC), como lo es Raspberry Pi,  siendo esta una unidad central o unidad de mando, permitiendo el control de la iluminación en la casa. También, ha sido implementado con diferentes módulos de comunicación, como ZigBee y WiFi, para conectarse a tarjetas Arduino Uno como hizo Kasmi \cite{Kasmi2016} y tarjetas Arduino Mega como realizo Tang \cite{Tang2017}. De igual modo, Kaneko \cite{Kaneko2017} ha empleado este mini computador como servidor local y controlador, adicionando a éste el módulo IRKit, para ser usado como control remoto de diferentes dispositivos por medio de IR.\\

Las tarjetas Arduino, además de ser usadas como dispositivos esclavos o secundarios, se usan como unidad de procesamiento central, en aplicaciones que no requiere un alto nivel de procesamiento. Para facilitar la comunicación de un Arduino Uno con diferentes dispositivos, Deshmukh \cite{Deshmukh2016} ha configurado un módulo bluetooth en el sistema, permitiendo así enviar datos de monitoreo y controlar una carga LED para iluminación. Así como también Howedi \cite{Howedi2016} lo usa para procesamiento de datos y control de actuadores. Verma \cite{Verma2016} ha implementado un módulo WiFi junto a esta placa, para ampliar su conexión a más dispositivos y enlazarlo directamente a Internet de las cosas (IoT). Por otra parte, Kusriyanto \cite{Kusriyanto2016} usa un módulo ethernet con Arduino Mega 2560 para ser el encargado de procesar y enviar datos a un servidor.\\

Sin embargo, no solo se usan tarjetas de prototipado, también se construyen nuevas tarjetas con funciones más específicas, así como Kusriyanto \cite{Kusriyanto2015}, el cual usó otro microcontrolador con mas pines como lo es el ATmega16, por lo cual este es otro modo de hacer eficiente el uso del hardware. Por otra parte, Sysala \cite{Sysala2016} ha importado de la industria el controlador lógico programable (PLC) para conectarlo en un entorno Smart House; esto debido a su gran estabilidad y robustez, en cuanto a un trabajo continuo y de potencia.\\

En Smart House, se ha implementado variedad de software, usado para la comunicación entre dispositivos móviles y el dispositivo central, más aun, que sea posible controlar la casa o realizar la comunicación entre el dispositivo central y los dispositivos esclavos. Del mismo modo, para ejecutar diferentes tareas como enviar datos al servidor, entre otras.\\

Owada \cite{Owada2012} ha usado HTML5 para desarrollar una aplicación dirigida a dispositivos con sistema operativo (OS) Android, esta se basa en un ambiente de novela gráfica, que permite controlar las diferentes partes de la casa. Así como una interfaz de usuario en un servidor web basado en JavaScript, que a su vez está vinculada con el sistema de control de la casa tal como desarrolla Behan \cite{Behan2013}.\\

El sistema Android es muy popular para probar las diferentes interfaces que se han creado, ya que es flexible en cuanto a la compatibilidad con estas, por ejemplo, permitiendo el acceso a un servidor web, que posee una aplicación basada en PHP y SQL desarrollada por Tseng \cite{Tseng2014}. Otro software para esta tarea, es VisualBasic .NET (VB.Net) el cual implementó Kusriyanto \cite{Kusriyanto2015} para enlazar el servidor con una aplicación de Android. Igualmente, se utilizan diferentes herramientas, como lo es App Inventor, el cual usa Howedi \cite{Howedi2016} para construir estas aplicaciones en el sistema Android, y de esta manera, no solo acceder por medio de la web, sino que, conecta la aplicación directamente con la unidad central, por medio de algún protocolo de comunicación  como Bluetooth. \\

Así, por ejemplo, Cheuque \cite{Cheuque2015} ha desarrollado una aplicación basadas en PHP, usando servidores Web como Lighttpd, el cual, tiene como soporte PostgreSQL para las bases de datos; esta aplicación se conecta a la unidad central de procesamiento con el fin de monitorear y controlar cargas LED; teniendo esto en cuenta, realizar aplicaciones en PHP es muy usado para controlar la casa, sea localmente o desde internet como realizo Kasmi \cite{Kasmi2016}. Otra alternativa implementada por Verma \cite{Verma2016}, es acoplar PHP con los servicios de MySQL, para desarrollar una página web HTML, que permita el control y monitoreo de la casa. Otro servidor externo, como Heroku, el cual fue usado por Kaneko \cite{Kaneko2017} para la visualización de datos desde cualquier lugar, sin necesidad de tener el servidor local compartido a internet. Esto se implementa por medio del lenguaje javascript, para interpretar diferentes comandos que se envían o se reciben. También, se evidencia el diseño y la visualización 3D de la casa, que será convertida en un entorno inteligente o simplemente un prototipo de esta por medio de Sweet Home 3D desarrollado por Howedi \cite{Howedi2016}.\\

%Empresas como Microsoft, IBM, entre otras, brindan servicios para IoT, como redes neuronales para identificar distintas caracteristicas sea en imagenes, en reconocimiento de voz y otros. Estas empresas ofrecen servicios gratuitos pero también de pago.