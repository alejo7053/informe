\chapter{Desarrollo e Implementación}

\section{Hardware}

El hardware se diseña por medio del software Proteus ...

\section{Firmware}

El firmware se desarrolla sobre el framework o SDK oficial de Espressif Systems, ESP-IDF el cual posee una documentación \cite{ES}, la cuál es muy útil a la hora de utilizar las diferentes APIs que este posee.\\

Sobre el firmware se desarrollan los siguientes temas:

\subsubsection{Consola}

Se usa para hacer pequeñas pruebas por medio de la construcción de comandos ligados a funciones ...

\subsubsection{HTTP Request}

Para realizar las diferentes peticiones de HTTP se utiliza la libreria lwip ...

\subsubsection{Tareas}

Las tareas son funciones propias de los sistemas operativos en tiempo real ...

\subsubsection{Timers}

Son utilizados principalmente para el control de cargas AC ...

\subsubsection{I2C}

El protocolo I2C se activa por medio de la instalación del driver en algún par de pines ...

\subsubsection{PWM}

Se usan para el control de cargas DC ...

\subsubsection{GPIO}

El ESP-WROOM-32 posee diferentes GPIO, los cuales se usan para leer o escribir señales digitales a los sensores o ...

\subsubsection{Interrupciones}

Las interrupciones se usan para no gastar recursos en un monitoreo constante de las entradas, solo cuando exista un cambio de nivel en la entrada el dispositivo desencadena una serie de instrucciones ...

\section{Software}

\subsection{Servidor Heroku}

\subsection{Framework Laravel}
