\chapter{Introducción}

 A lo largo del crecimiento de los entornos inteligentes, como Smart House, se han realizado investigaciones con múltiples orientaciones, las cuales están enfocadas en razones sociales como la comodidad y la seguridad, sin dejar de lado factores ambientales como el ahorro energético. En cuanto a una parte más técnica, estos procesos inteligentes están compuestos por software, hardware y firmware.\\
 
 Las investigaciones hacia el entorno de Smart House se enfocan en monitorear y/o controlar múltiples aspectos de una casa. Para realizar esta tarea físicamente se usa un hardware, en el cual se ven inmersos la unidad central de procesamiento, los sensores y los actuadores, la primera se encarga del monitoreo y control del entorno.\\
 
 Autores como Behan \cite{Behan2013} y Cheuque \cite{Cheuque2015} han usado mini computadoras o computadoras de placa simple (SBC), como lo es Raspberry Pi, siendo esta una unidad central o unidad de mando, permitiendo el control de la iluminación en la casa. Sin embargo, no solo se usan tarjetas de prototipado, también se construyen nuevos dispositivos con funciones más específicas, así como Kusriyanto \cite{Kusriyanto2015}, el cual usó un microcontrolador ATmega 16, el cual cuenta con más pines, por lo cual este es otro modo de hacer eficiente el uso del hardware.\\
 
 En Smart House, se ha implementado variedad de software, usado para la conexión entre los dispositivos móviles y el central, más aun, que sea posible controlar la casa o realizar la comunicación entre el dispositivo central y los esclavos. Del mismo modo, para ejecutar diferentes tareas como enviar datos al servidor, entre otras.\\
 
 Así, por ejemplo, Cheuque \cite{Cheuque2015} ha desarrollado una aplicación basada en PHP, usando servidores Web como Lighttpd, el cual tiene como soporte PostgreSQL para las bases de datos; esta aplicación se conecta a la unidad central de procesamiento con el fin de monitorear y controlar cargas LED; teniendo esto en cuenta, realizar aplicaciones en PHP es muy usado para controlar la casa, sea localmente o desde internet como realizo Kasmi \cite{Kasmi2016}. Otro servidor externo, como Heroku, el cual fue usado por Kaneko \cite{Kaneko2017} para la visualización de datos desde cualquier lugar, sin necesidad de tener el servidor local compartido a internet.\\
 
 Los dispositivos inteligentes aumentan a gran velocidad, por la necesidad de estar siempre conectados, dando paso a aplicar esta conexión a diferentes espacios del hogar; una casa inteligente o Smart House se compone de multiples dispositivos que se encuentran conectados a la red con posibilidad de acceso desde cualquier parte del mundo. \\
 
 En este trabajo se realiza la construcción completa de una solución para Smart House, desarrollando tanto el firmware como el hardware y el software. En cuanto al hardware, este cuenta con múltiples entradas para monitorear el entorno de aplicación por medio de sensores, así como también cuenta con salidas enfocadas a cargas AC y DC de corriente directa, en busca de gestionar y controlar dicho entorno. El software se ve reflejado en la construcción de una aplicación web, cuya característica principal es el panel de control, donde se muestran los estados y valores de los sensores así como también los estados de las cargas junto con su correspondiente control de manera simple para el usuario, de tal manera que por medio de firmware e internet se vinculen las interacciones generadas y recibidas en el software con su correspondiente carga o sensor en el hardware.\\
 
 Este trabajo está organizado en siete capítulos. El lector en el capítulo 2 encontrará la descripción del objetivo general y objetivos específicos de este trabajo. En el capítulo 3 se encuentra recopilado el marco teórico, distribuido en conceptos importantes de IoT, así como también del Hardware y el Software en cuestión. El capítulo 4 presenta el desarrollo, pasando por la construcción del hardware y la creación del firmware y software. Los resultados de este trabajo se encuentran plasmados en el capítulo 5. En el capítulo 6 se muestran las conclusiones y en el capítulo 7, 8 y 9 se encuentran los trabajos futuros, glosario y la bibliografía respectivamente, por último se presentan los Anexos.\\
 
