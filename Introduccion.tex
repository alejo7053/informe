\chapter{Introducción}

 A lo largo del crecimiento de los entornos inteligentes, como Smart House, se han realizado investigaciones con múltiples orientaciones, las cuales están enfocadas en razones sociales como la comodidad y la seguridad, sin dejar de lado factores ambientales como el ahorro energético. En cuanto a una parte más técnica, estos procesos inteligentes se componen por software, hardware y firmware.\\
 
 Las investigaciones hacia el ámbito de Smart House se enfocan en monitorear y/o controlar múltiples aspectos de una casa. Para realizar esta tarea físicamente se usa un hardware, en el cual se ven inmersos la unidad central de procesamiento, los sensores y los actuadores, la primera se encarga del monitoreo y control del entorno.\\
 
 Autores como Behan \cite{Behan2013} y Cheuque \cite{Cheuque2015} han usado mini computadoras o computadoras de placa simple (SBC), como lo es Raspberry Pi, siendo esta una unidad central o unidad de mando, permitiendo el control de la iluminación en la casa. Sin embargo, no solo se usan tarjetas de prototipado, también se construyen nuevos dispositivos con funciones más específicas, así como Kusriyanto \cite{Kusriyanto2015}, el cual usó un microcontrolador ATmega 16, el cual cuenta con más pines, por lo cual este es otro modo de hacer eficiente el uso del hardware.\\
 
 En Smart House, se ha implementado variedad de software, usado para la conexión entre los dispositivos móviles y el central, más aun, que sea posible controlar la casa o realizar la comunicación entre el dispositivo central y los esclavos. Del mismo modo, con el fin de ejecutar diferentes tareas como enviar datos al servidor, entre otras.\\
 
 Así, por ejemplo, Cheuque \cite{Cheuque2015} ha desarrollado una aplicación basada en PHP, usando servidores Web como Lighttpd, el cual se soporta en PostgreSQL para las bases de datos; esta aplicación se conecta a la unidad central de procesamiento con el fin de monitorear y controlar cargas LED; teniendo esto en cuenta, realizar aplicaciones en PHP es muy usado a fin de controlar la casa, sea localmente o desde la web como realizo Kasmi \cite{Kasmi2016}. Otro servidor externo, Heroku, el cual fue usado por Kaneko \cite{Kaneko2017} para la visualización de datos desde cualquier lugar, sin necesidad de tener el servidor local compartido a internet.\\
 
 Los dispositivos inteligentes aumentan a gran velocidad, por la necesidad de estar siempre conectados, dando paso a aplicar esta conexión a diferentes espacios del hogar; una casa inteligente o Smart House se compone de multiples artefactos que se encuentran enlazados a la red con posibilidad de acceso desde cualquier parte del mundo. \\
 
 En este trabajo se realiza la construcción completa de una solución para Smart House, desarrollando el hardware, firmware y software. El hardware cuenta con múltiples entradas con el fin de monitorear el entorno de aplicación por medio de sensores, también posee salidas enfocadas a cargas AC y DC, en busca de gestionar y controlar dicho ámbiente. El software se ve reflejado en el desarrollo de una aplicación web, cuya característica principal es el panel de control, donde se muestran los valores de los sensores y asimismo los estados de las cargas junto con su correspondiente control de forma simple para el usuario, de tal manera que a través del firmware e internet se vinculen las interacciones generadas y recibidas en el software con su respectiva carga o sensor en el hardware.\\
 
 Este trabajo está organizado en siete capítulos. El lector en el capítulo 2 encontrará la descripción del objetivo general y objetivos específicos. En el capítulo 3 se encuentra recopilado el marco teórico, distribuido en conceptos importantes de IoT, así como también del Hardware y el Software en cuestión. El capítulo 4 presenta el desarrollo, pasando por la construcción del hardware y la creación del firmware y software. Los resultados se encuentran plasmados en el capítulo 5. En el capítulo 6 se muestran las conclusiones y en el capítulo 7 están los trabajos futuros, por último el glosario, la bibliografía y los Anexos.\\
 
