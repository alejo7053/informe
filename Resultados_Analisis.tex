\chapter{Resultados y Análisis}

Para este capitulo se propone una habitación modelo, con la cual se describen ciertos pasos y funcionamientos de la solución en general. En la figura \ref{fig:iot} esta el esquema de la solución IoT, para esto se supone una habitación con un sensor de temperatura, un ventilador y un bombillo led, las cuales el usuario va a visualizar y gestionar desde la aplicación web.

\begin{figure}[H]
	\centering
	\caption{Esquema Solución SmartHouse [Imagen Propia]}
	\label{fig:iot}
	\includegraphics[width=0.6\linewidth]{Imagenes/IOT}
\end{figure}

\section{Software}

Se desarrolla la aplicación web de manera local y posteriormente se lanza a un servidor en Internet. Se encuentra compuesta por los siguientes sitios y las diferentes interacciones basadas en las funciones básicas, crear, leer, actualizar y borrar (CRUD).

\begin{itemize}
	\item Parte Pública
	\item Parte Privada
	\begin{itemize}
		\item API
		\item Panel de Control
		\begin{itemize}
			\item Crear
			\item Ver
			\item Editar
			\item Eliminar 
		\end{itemize}
	\end{itemize}
\end{itemize}

De acuerdo a la lista anterior, se toman en cuenta dos partes para esta, una pública y una privada, como se observa en la figura \ref{fig:index}. En la parte pública se encuentra una vista con los datos de contacto, solicitudes de registro o productos y la cantidad de usuarios que actualmente estan registrados en la aplicación. En la parte privada se encuentra la interacción de los usuarios sea administrador, dueño de una casa o de una habitación, para controlar y ver sus datos.\\

Las diferentes interacciones que tiene cada usuario en el panel de control se garantizan por medio del framework, creando diferentes roles para cada usuario que se esta registrando y realizando la comprobación por parte de los controladores y el middleware que este provee.\\

\begin{figure}[H]
\centering
\caption{Página de Inicio. [Imagen Propia]}
\label{fig:index}
\includegraphics[width=0.5\linewidth]{Imagenes/Index}
\end{figure}

\subsection{Parte Pública}

En esta vista unicamente hay opciones para el contacto y solicitudes, como se menciona anteriormente, es una vista sencilla dada la poca información que contiene, como se observa en la figura \ref{fig:publicview}.

\begin{figure}[H]
\centering
\caption{Vista Pública. [Imagen Propia]}
\label{fig:publicview}
\includegraphics[width=0.5\linewidth]{Imagenes/Public_view}
\end{figure}

\subsection{Parte Privada}

En esta sección es donde se encuentra el Panel de Control para los diferentes usuarios de la aplicación. En primera instancia, para un usuario administrador, que es el encargado de gestionar la aplicación, este usuario tiene la posibilidad de crear, ver, editar y eliminar los diferentes registros de la aplicación, la vista de este usuario se puede observar en la figura \ref{fig:views}\textbf{(a)}. Por medio de este usuario es que se activan las cuentas de los demás, por esto en la parte pública estás las opciones de contacto y solicitud de registro.\\

También existe el usuario dueño de la casa donde se encuentra el dispositivo, este usuario es opcional y es para gestionar los dispositivos presentes dentro de una misma casa, es un administrador de la casa, el cual puede ver y editar algunos campos de sus usuarios hijos o usuarios habitación y sus diferentes casas y habitaciones, unicamente las que esten registradas a su nombre, como se ve en la figura \ref{fig:views}\textbf{(b)}, de este modo el rol de este usuario es administrar su casa y visualizar los datos de esta.\\

Por último, otro rol es el de usuario habitación, el cuál es un usuario que solo visualiza sus propios datos, como la habitación y los dispositivos presentes en esta, como se observa en la figura \ref{fig:views}\textbf{(c)}, a este solo le compete la información de lo que posee en su habitación, por tal motivo el panel de control muestra una vista general de los datos y el estado de sus dispositivos, además de tener la capacidad de editar partes básicas de su habitación y perfil. Este usuario puede o no estar sujeto a un usuario padre o usuario casa, ya que, solamente puede poseer una tarjeta para su habitación y ninguna otra en dicha casa.\\

Continuando con este usuario y la habitación modelo propuesta al inicio del capitulo, luego de que el usuario accede a la aplicación web e inicia sesión con los datos que ha registrado en el sistema, al ser un usuario de una habitación, este se encuentra con un panel de control como el de la figura \ref{fig:views}\textbf{(c)}, allí puede gestionar los dispositivos presentes en su habitación, así pues, puede visualizar la temperatura que se ha sensado y también encender o apagar los dispositivos conectados a la tarjeta.\\

\begin{figure}[H]
	\centering
	\caption{Vistas de Usuarios [Imagen Propia]}
	\label{fig:views}
	\subfigure[Usuario Administrador]{\includegraphics[width=0.9\linewidth]{Imagenes/Admin_view}}
	\subfigure[Usuario de Casa]{\includegraphics[width=0.45\linewidth]{Imagenes/UserH_view}}
	\subfigure[Usuario de Habitación]{\includegraphics[width=0.45\linewidth]{Imagenes/UserR_view}}
\end{figure}

Además de este panel de control, desde el cuál se realizan las operaciones sobre la aplicación, en la parte privada se encuentra la ruta encargada de la actualización Servidor-Tarjeta, es decir, en esta ruta es donde se da la comunicación. En esta ruta se realiza una petición HTTP tipo GET por parte de la tarjeta, esta contiene el id de la habitación en la cuál esta instalada la tarjeta y también el token correspondiente a esta, y además en la URL se añade un texto tipo JSON en el cual se encuentra toda la información de la lectura actual de los sensores, esta petición la responde el servidor con un texto, también tipo JSON que contiene la información pertinente de las cargas o actuadores, como se observa la figura \ref{fig:updateview}, para dar seguridad a esta transacción, se se utiliza el token mencionado anteriormente, el cual se verifica mediante el id de la habitación y que este coincida con los datos almacenados, de este modo se garantiza que lo que se envía este sometido a verificación.\\

\begin{figure}[H]
	\centering
	\caption{Página de intercambio de datos [Imagen Propia]}
	\label{fig:updateview}
	\includegraphics[width=0.7\linewidth]{Imagenes/Update_view}
\end{figure}

De este modo, el usuario interactuando con la aplicación genera modificaciones en el texto con que responde la aplicación web a la petición de la tarjeta. Si el usuario desea encender el ventilador, como se observa en la figura \ref{fig:userhview} esta presente un botón en la información del dispositivo, el cual con presionarlo lo enciende o apaga, esto es valido para cualquiera de los dos dispositivos, sea el ventilador o el bombillo led.\\

Si el ventilador esta conectado a una salida de AC controlada es posible que por medio del deslizador se le asigne un valor para que cambie su funcionamiento, del mismo modo para el bombillo led donde se refleja en su cambio de intensidad, pero este conectado a una salida DC controlada. Al generar estas interacciones el texto en formato JSON cambia de acuerdo a lo pedido por el usuario.\\ 

También si el usuario desea añadir, modificar o eliminar una regla, por ejemplo, desea encender el bombillo led a una hora deseada, esto lo puede lograr mediante el botón de reglas en el panel de control, el cual lo redirige a la vista que se observa en la figura \ref{fig:rulesview}, en esta se indica una hora de inicio y finalización en la cual el bombillo led enciende a la hora de inicio y se apaga a la hora de fin, si es la regla de apagado solo necesita una hora de inicio para apagar el bombillo.

\begin{figure}[H]
	\centering
	\caption{Vista para añadir reglas [Imagen Propia]}
	\label{fig:rulesview}
	\includegraphics[width=0.6\linewidth]{Imagenes/rules_view}
\end{figure}

\subsection{Base de Datos}

La estructura de la base de datos se puede observar en la figura \ref{fig:db}, aquí se observan los diferentes campos que posee cada tabla, además de las llaves y sus relaciones, las relaciones presentes en esta estructura son de tipo 1:N, es decir, por ejemplo un usuario puede tener relacionadas N casas.

\begin{figure}[H]
\centering
\caption{Base de datos SmartHouse [Imagen Propia]}
\label{fig:db}
\includegraphics[width=0.7\linewidth]{Imagenes/DB}
\end{figure}

\section{Firmware}

El firmware se encuentra compuesto, como se ha mencionado anteriormente, de tareas, en la figura \ref{fig:tareas} se observa un bosquejo de como funcionan las diferentes tareas de las que se compone este, tomando como función principal la encargada de gestionar la conexión a Wi-Fi y almacenar sus credenciales, dependiendo del estado de si encuentra o no estas, el sistema se comporta de una u otra forma. Si existen credenciales almacenadas en la tarjeta, el sistema se trata de conectar, si la conexión es exitosa comienza el proceso de actualización de la hora del sistema y también de las diferentes ordenes de la aplicación web. En cambio si no existen credenciales el dispositivo inicia un servidor web local para que el usuario pueda proporcionarle estas credenciales como se menciona en la sección \ref{sub:wifi}.\\

En cada caso se crean tareas diferentes, para el caso de no existir credenciales únicamente se configura la tarea del servidor http local, y para el otro caso se lanzan todas la demás tareas encargadas de la escritura y lectura de datos, como se menciona en las siguientes secciones. Cabe resaltar que la tarjeta tiene disponible en promedio 160KB de memoria heap para realizar otras operaciones o implementar más funcionalidades en esta.

\begin{figure}[H]
	\centering
	\caption{Esquema de Tareas [Imagen Propia]}
	\label{fig:tareas}
	\includegraphics[width=0.7\linewidth]{Imagenes/tareas}
\end{figure}

Además de esto, se mide el promedio del tiempo en que se demora la ejecución de la tarea, en dos casos, cuando realiza la primera petición después de conectado a la red, es decir, cuánto se demora realizando las peticiones necesarias, como obtener el tiempo de la red y realizar la petición a la aplicación web, obteniendo un tiempo promedio de aproximadamente de 2.7s, este tiempo depende de la disponibilidad de los diferentes servidores en la web además de la velocidad y el tráfico de la red, ya que la tarjeta espera hasta que obtiene la hora y luego continua con la petición HTTP también esperando que el servidor responda. El otro caso es el tiempo que se demora la tarea en leer los datos de los sensores, enviar la petición HTTP a la aplicación web, recibirlos y enviarlos a los diferentes actuadores, este tiempo es de 1 s en promedio con un total de 3056 muestras.

\subsection{Conexión a Internet vía Wi-Fi}\label{sub:wifi}

Los sistemas IoT deben estar conectados siempre a Internet, por este motivo se debe brindar una forma para conectar al sistema a este, por lo tanto, se desarrolla un servidor local en la tarjeta que se encarga de esto, como se ha mencionado el módulo del esp32 funciona como Punto de Acceso (AP) y como Cliente o Estación (STA) al mismo tiempo, aprovechando esta capacidad se usa el servidor local y esta encargado de gestionar la conexión de la tarjeta vía Wi-Fi como se observa en la figura \ref{fig:conexion}.\\

\begin{figure}[H]
	\centering
	\caption{Conexión a Internet vía Wi-Fi ESP32 [Imagen Propia]}
	\label{fig:conexion}
	\includegraphics[width=0.7\linewidth]{Imagenes/conexion}
\end{figure}


De este modo, en la figura \ref{fig:wifi} están algunas paginas del servidor local de la tarjeta, en la figura \ref{fig:red} se puede ver la lista de las diferentes redes al alcance de la tarjeta, basta con seleccionar una red e ingresar sus credenciales para conectarse a esta, en la figura \ref{fig:opt} se observan los detalles de la conexión actual y también la opción de desconectarse de esta. Su funcionamiento es muy intuitivo, se selecciona la red a la que se desea conectar la tarjeta, se ingresan sus credenciales y posteriormente el dispositivo verifica si la conexión fue exitosa o no, si la conexión es exitosa ya la tarjeta esta lista para su funcionamiento, se debe reiniciar para que solo quede funcionando como STA y no en el modo dual, además de esto si las credenciales de la red cambian también se incluye un botón para el borrado de estas, para que se puede configurar de nuevo la conexión a la red Wi-Fi.

\begin{figure}[H]
	\centering
	\caption{Aplicación Conexión a Wi-Fi [Imagen Propia]}
	\label{fig:wifi}
	\subfigure[Lista de Redes]{\includegraphics[width=0.45\linewidth]{Imagenes/w_status}
		\label{fig:red}}
	\subfigure[Datos de Conexión]{\includegraphics[width=0.45\linewidth]{Imagenes/w_red}
		\label{fig:opt}}
\end{figure}

\subsection{Escritura de Datos en la Aplicación Web}

Los datos que esta leyendo la tarjeta provienen de los diferentes sensores que tiene conectados como se ha mencionado, se usan diferentes tipos, como de estado para sensar la presencia, de calidad de aire entre otros presentes en esta. Para la escritura de los datos, en el firmware, se desarrollan diferentes tareas encargadas de leer y enviar estos a una tarea central. Los datos que están enviando contienen el id del dispositivo y la medida que lee en ese momento, estos se envían en forma de texto en formato JSON, de este modo, la tarea central los gestiona y envía a la aplicación con el mismo formato, organizandolos en la petición HTTP tipo GET que realiza, así la url que la tarjeta solicita, incluyendo el JSON de cada sensor, se observa en la figura \ref{fig:json}, de este modo, en la url se encuentra el dominio del servidor, y la dirección que contiene el id y token de la habitación, por último esta la información del sensor en formato JSON, que se compone por el id y el valor de este. La aplicación ya se encarga de almacenarlos y mostrarlos al usuario como se menciona anteriormente.

\begin{figure}[H]
	\centering
	\caption{URL de la petición HTTP [Imagen Propia]}
	\label{fig:json}
	\includegraphics[width=0.7\linewidth]{Imagenes/JSON}
\end{figure}


\subsection{Lectura de Datos de Internet}

Para la configuración y comparación de las reglas que suministra el usuario a el dispositivo que dese controlar es necesario contar con la hora actual y que se siga actualizando localmente gracias al RTC que posee internamente el esp32, así, al inicio de la aplicación, se sincroniza y almacena la hora actual de la red, por medio del protocolo SNTP. Estas reglas actúan en cuanto a que encienda o apague un dispositivo a una hora dada, después de obtener este dato la aplicación continua con normalidad para realizar las diferentes peticiones a la aplicación web.\\

La interacción del usuario se da con la aplicación web mencionada anteriormente, de este modo la tarjeta siempre se debe actualizar, para esto, cuando la tarjeta envía los datos de los sensores la aplicación responde con datos de cabecera HTTP y además la información de los dispositivos que controla la tarjeta, esta los recibe en una cadena texto en formato JSON como se observa en la figura \ref{fig:httprqstesp}, los procesa y envía a las tareas pertinentes ya sea para encender o apagar algún dispositivo conectado a la tarjeta, además envía las reglas que el usuario ha definido.

\begin{figure}[H]
	\centering
	\caption{Respues del la APP Web a la Tarjeta [Imagen Propia]}
	\label{fig:httprqstesp}
	\includegraphics[width=0.8\linewidth]{Imagenes/HTTPRqstesp}
\end{figure}


\section{Hardware}

De acuerdo a los circuitos diseñados en la sección \ref{sec:hw} donde se propone el desarrollo de hardware de la solución IoT, en la figura \ref{fig:tarjeta} se observan las diferentes tarjetas ya ensambladas en una caja eléctrica para probar el funcionamiento del prototipo. Las salidas y entradas están están distribuidas por la caja eléctrica de acuerdo a lo propuesto, para las salidas AC se usan toma corrientes para conectar allí los diferentes dispositivos, para las salidas DC se utilizan conectores hembra tipo banana para facilitar la conexión de estos dispositivos.\\

\begin{figure}[H]
	\centering
	\caption{Tarjeta SmartHouse [Imagen Propia]}
	\label{fig:tarjeta}
	\includegraphics[width=0.6\linewidth]{Imagenes/Tarjeta}
\end{figure}


La distribución de las diferentes salidas que posee la tarjeta se organizo en pegatinas y se colocaron sobre la caja eléctrica, la figura \ref{fig:labels} muestra esta información, tanto para la parte interna como externa de la caja.\\

\begin{figure}
	\centering
	\caption{Descripción caja eléctrica tarjeta SmartHouse [Imagen Propia]}
	\label{fig:labels}
	\includegraphics[width=0.7\linewidth]{Imagenes/labels}
\end{figure}


De acuerdo a lo mencionado anteriormente, el sistema se ha probado con cargas AC como bombillos LED entre 7W y 20W, de filamento de 100W, probando funcionalidades como el control de potencia AC por ángulo de fase, obteniendo los resultados esperados, como se observa en la figura \ref{fig:ACc}\textbf{(a)} donde la carga tiene el 100\% de la potencia y en la figura \ref{fig:ACc}\textbf{(b)} con el 20\% de esta. El voltaje de alimentación viene dado por la red eléctrica, la tarjeta simplemente conmuta el estado de la alimentación o controla la potencia entregada. 

\begin{figure}[H]
	\centering
	\caption{Control de potencia AC por ángulo de fase [Imagen Propia]}
	\label{fig:ACc}
	\subfigure[Potencia al 100\%]{\includegraphics[width=0.45\linewidth]{Imagenes/AC1}}
	\subfigure[Potencia al 20\%]{\includegraphics[width=0.43\linewidth]{Imagenes/AC0}}
\end{figure}

Para las salidas DC se realizan pruebas con un motor DC de 1W, además de una tira LED de 1W, la cual se le varia la energía entrega, como se observa en la figura \ref{fig:DCc}\textbf{(a)} la carga tiene el 100\% de la energía y en la figura \ref{fig:DCc}\textbf{(b)} solamente el 10\%. Estas cargas se alimentan con 12VDC directamente desde la fuente o convertidor AC-DC, los circuitos que se implementan simplemente conmutan el estado de encendido/apagado o por medio de PWM variar la energía entregada.

\begin{figure}[H]
	\centering
	\caption{Control de Cargas DC [Imagen Propia]}
	\label{fig:DCc}
	\subfigure[Potencia al 100\%]{\includegraphics[width=0.45\linewidth]{Imagenes/DC1}}
	\subfigure[Potencial al 10\%]{\includegraphics[width=0.455\linewidth]{Imagenes/DC0}}
\end{figure}

\section{Prueba Beta Cerrada}

Para la prueba beta se escoge un grupo de personas, las cuales interactuan directamente con la aplicación web y el prototipo de la tarjeta SmartHouse, se detallan los diferentes items a evaluar mencionados anteriormente, como el ingreso a la aplicación, visualización de los datos almacenados en esta y el control de los dispositivos.\\

Así, para evaluar estas características se formulan las siguientes preguntas, que se califican de acuerdo a una escala tipo likert \cite{lik} de uno a cinco en la cual, cinco es la calificación máxima y uno la mínima. Las preguntas se formulan para evidenciar la postura del cliente en cuanto a las funcionalidades de la aplicación, de acuerdo a esto la encuesta que se diseña esta presente en el Anexo \ref{AnexoB}.

Los resultados de la prueba realizada a cinco personas se consignan en la tabla \ref{table:enc}, realizando el promedio de cada pregunta y presentando un resultado total.

\begin{table}[H]
	\begin{center}
		\caption{Resultados por pregunta.}
		\label{table:enc}
		\begin{tabular}{|c|c|}
			\hline 
			Número de la Pregunta & Promedio \\ 
			\hline 
			1 & 4.4\\ 
			\hline 
			2 & 5\\ 
			\hline 
			3 & 4.6\\ 
			\hline 
			4 & 5\\ 
			\hline 
			5 & 5\\ 
			\hline 
			6 & 5\\ 
			\hline 
			7 & 5\\ 
			\hline 
			8 & 5\\ 
			\hline 
			9 & 4.6\\ 
			\hline 
			10 & 5\\ 
			\hline 
			11 & 4.8\\ 
			\hline 
		\end{tabular} 
	\end{center}
\end{table}

En general la aplicación recibe una calificación de 4.9, por lo tanto se puede decir que las funcionalidades requeridas están programadas de una manera adecuada y simple para que el usuario disponga de ellas.