\chapter{Glosario}

\textbf{AC (Corriente alterna):} corriente eléctrica variable en la que las cargas eléctricas (electrones) cambian el sentido del movimiento a través de un conductor de manera periódica.\\

\textbf{DC (Corriente continua):} corriente de intensidad constante en la que el movimiento de las cargas eléctricas (electrones) siempre es en el mismo sentido.\\

\textbf{Internet del todo (IoE):} es un concepto que extiende el énfasis de la internet de las cosas (IoT) en las comunicaciones de máquina a máquina para describir un sistema más complejo que también abarca personas y procesos.\cite{IOE} \\

\textbf{Internet de las Cosas (IoT):} parte fundamental del internet del todo (IdT), el cual se refiere principalmente a la interacción máquina-máquina, en incluso interacción máquina-persona.\\

\textbf{Software:} conjunto de programas y rutinas que permiten a un sistema realizar determinadas tareas.\\

\textbf{Hardware:} partes físicas que componen un sistema electrónico, como por ejemplo los componentes de un circuito electrónico.\\

\textbf{Firmware:} programa informático que establece la lógica de más bajo nivel que controla los circuitos electrónicos de un dispositivo de cualquier tipo, es decir, software que maneja físicamente al hardware.\\

\textbf{Radiofrecuencia:} es la porción del espectro electromagnético (frecuencias) que es empleado en la radiocomunicación.\\

\textbf{Infrarrojo:} se refiere a la radiación electromagnética con longitud de onda mayor (menor energía) a la de la luz visible por el ser humano.\\

\textbf{PIC:} familia de microcontroladores tipo RISC (Computador con Conjunto de Instrucciones Reducidas). \cite{PIC}\\

\textbf{Zigbee:} es el nombre de la especificación de un conjunto de protocolos de alto nivel de comunicación inalámbrica para su utilización con radiodifusión digital de bajo consumo, basada en el estándar IEEE 802.15.4 de redes inalámbricas de área personal. \cite{ZB}\\

\textbf{HTML5:} siendo la última versión de HTML, contiene elementos, atributos y comportamientos nuevos, ademas de un conjunto más amplio de tecnologías que proporciona mayor diversidad y alcance a los sitios Web. \\

\textbf{PHP (Preprocesador de hipertexto):} es un lenguaje de código abierto muy popular especialmente adecuado para el desarrollo web y que puede ser incrustado en HTML.\\

\textbf{SQL (lenguaje de consulta estructurado):} es un lenguaje de programación estándar e interactivo para la obtención de información desde una base de datos y para actualizarla.\\

\textbf{VisualBasic .NET:} es un lenguaje de programación orientado a objetos que cuenta con los beneficios que le brinda .NET Framework, el modelo de programación diseñado para simplificar la programación de aplicaciones en un entorno sumamente distribuido como lo es Internet.\cite{VB}\\