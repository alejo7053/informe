\chapter{Marco Teórico}

\section{Internet de las Cosas}

La internet de las cosas es un sistema de dispositivos de computación interrelacionados, máquinas mecánicas y digitales, objetos, animales o personas que tienen identificadores únicos y la capacidad de transferir datos a través de una red, sin requerir de interacciones humano a humano o humano a computadora. \\

IoT ha evolucionado desde la convergencia de tecnologías inalámbricas, sistemas micro-electromecánicos, microservicios e internet. La convergencia ha ayudado a derribar las paredes de silos entre la tecnología operativa  y la tecnología de la información, permitiendo que los datos no estructurados generados por máquinas sean analizados para obtener información que impulse mejoras. \cite{TechT2017}\\

Kevin Ashton, cofundador y director ejecutivo del Auto-ID Center de MIT, mencionó por primera vez la internet de las cosas en una presentación que hizo a Procter \& Gamble en 1999. He aquí cómo Ashton explica el potencial de la internet de las cosas:

``Las computadoras de hoy –y, por lo tanto, la internet– dependen casi totalmente de los seres humanos para obtener información. Casi todos los aproximadamente 50 petabytes (un petabyte son 1.024 terabytes) de datos disponibles en internet fueron capturados y creados por seres humanos escribiendo, presionando un botón de grabación, tomando una imagen digital o escaneando un código de barras. \\

El problema es que la gente tiene tiempo, atención y precisión limitados, lo que significa que no son muy buenos para capturar datos sobre cosas en el mundo real. Si tuviéramos computadoras que supieran todo lo que hay que saber acerca de las cosas –utilizando datos que recopilaron sin ninguna ayuda de nosotros– podríamos rastrear y contar todo, y reducir en gran medida los desechos, las pérdidas y el costo. Sabríamos cuándo necesitamos reemplazar, reparar o recordar cosas, y si eran frescas o ya pasadas”. \cite{Asthon2009}\\

\section{Smart House}

El concepto de Smart House implica tres características básicas. En primer lugar, el monitoreo a través de redes de sensores para obtener información sobre la casa y sus residentes. En segundo lugar, los mecanismos que controlan el uso de la comunicación entre dispositivos para permitir la automatización y el acceso remoto. Por último, las interfaces de usuario, como los teléfonos inteligentes y las computadoras que permiten a los usuarios especificar las preferencias, así como presentar información a las personas acerca de estas preferencias. \\

Smart House es un entorno que tiene sistemas sofisticados a través de los cuales se pueden controlar algunas de las cosas de la casa, como luces, puertas, ventanas, además  puede racionalizar el consumo de energía, entre otras funciones mediante el uso de sensores. Básicamente, uno de los beneficios más importantes del uso de la tecnología en las casas, es la prestación de servicios a las personas.\cite{Howedi2016} \\

\section{Hardware}

\subsection{ESP32}

\subsection{Detector de Cruce por cero}

\section{Software}

\subsection{RTOS}

\subsection{ESP-IDF}