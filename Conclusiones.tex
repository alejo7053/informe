\chapter{Conclusiones}

\begin{itemize}
	\item El sistema compuesto de Firmware, Hardware y Software descrito en este documento es una solución IoT funcional, ya que este cuenta con la capacidad de monitorear y controlar un entorno de aplicación por medio de internet, el cual es en este caso una habitación de Smart House, siendo esta la representación de uno de los infinitos escenarios posibles a los cuales puede ser conectado el sistema, puesto que está diseñado para abarcar una amplia capacidad de tareas, además de que acepta múltiples tipos de dispositivos de medida; combinando estos aspectos con una Paas que permite la interacción humano-maquina, de tal manera que cumple con las características principales para una solución IoT.\\
	
	\item El hardware implementado para la solución IoT, funciona en conjunto con el firmware que enlaza su parte física con el software presente en la nube; este fue diseñado en dos etapas con un circuito para cada una, las cuales son la etapa de potencia AC y la etapa DC. Esta organización del sistema no solo permite una clasificación en cuanto a su funcionamiento, sino que también evita que los altos voltajes de la etapa AC causen algún tipo de interferencia de manera directa con la etapa DC. Cada etapa del hardware cuenta con una alimentación que garantiza los niveles correctos para la alimentación de los sensores y cargas, además de una apropiada lectura de cada dispositivo de medida y una adecuada manipulación del entorno.\\
	
	\item La aplicación web presentada en este documento, permite la interacción del usuario con los diferentes dispositivos que se encuentran conectados al hardware. Esta caracteristica del sistema se encarga de la gestión de la información colectada por el hardware junto con las órdenes administradas por el mismo, ya sea por visualización o manipulación del entorno de aplicación, es decir, para su monitoreo o control, ya que se tiene acceso desde cualquier lugar con conexión a internet; de este modo la aplicación cumple con los diferentes alcances y puede seguir siendo ampliada para adicionar funcionalidades al sistema.\\
	
	\item La prueba Beta realizada para evaluar el sistema con la participación de los usuarios finales otorgo una calificación de 4.9, lo cual permite confirmar que se ha desarrollado de la mejor forma posible el sistema, desde el encendido y manipulación del hardware junto con sus conexiones, hasta la navegación a través de la aplicación web, facilitando estos procesos con ayuda de un manual de uso donde se encuentran indicados y detallados, ya que este fue escrito con la intención de ser entendido y seguido por cualquier usuario.\\
	
	\item La tarjeta de prototipado ESP32 como unidad central de procesamiento dentro del desarrollo del hardware, representa una opción viable en la implementación de un sistema IoT, no solo por su bajo costo sino también por sus características y funcionalidades, las cuales están diseñadas para este tipo de aplicaciones, permitiendo así que el diseño y la implementación del sistema sea escalable.\\
	
	\item El uso del framework ESP-IDF, que se compone de un sistema operativo en tiempo real freeRTOS, es muy útil para realizar las diferentes funciones del sistema que comunica hardware con software, además de que permite realizar las diferentes tareas y gestionar bien los recursos presentes en el chip ESP32, por lo tanto es posible agregar una gama más amplia de características a la funcionalidad del sistema.\\
	
\end{itemize}