z\chapter{Conclusiones}

\begin{itemize}
	\item El sistema compuesto de Firmware, Hardware y Software descrito en este documento, es una solución IoT funcional, ya que permite monitorear y controlar un entorno de aplicación por medio de internet, dicho entorno es en este caso una habitación de Smart House, siendo esta la representación de un escenario posible, entre infinidad de escenarios donde conectar este sistema.
	
	\item Se desarrolla un prototipo de una tarjeta inalámbrica, implementando el hardware como el firmware que enlaza la parte fisica con el software presente en la nube. Esta cumple con los requisitos de alimentación de los sensores y las cargas, además de la lectura adecuada de los sensores y el control de las diferentes cargas que se puedan conectar.
	
	\item La aplicación web presentada en este documento permite la interacción del usuario con los diferentes dispositivos que se encuentran conectados al hardware; esta se encarga de la gestión de la información producida por el hardware, ya sea por visualización o interacción del entorno de aplicación, es decir para su monitoreo o control, lo cual representan unas de características principales de las soluciones del internet de las cosas, ya que se tiene acceso desde cualquier lugar con conexión a internet, además de comprender el funcionamiento de las PaaS y cómo usarlas de una manera adecuada. De este modo la aplicación cumple con los diferentes alcances y se puede seguir ampliando para aumentar sus funcionalidades.
	
	\item La calificación de 4.9 resultante de la prueba Beta, es decir, la prueba realizada para evaluar el sistema completo con la participación de los usuarios finales, permite confirmar que se ha desarrollado de la mejor forma posible, pues con ayuda de un manual de uso, se presentan los datos de manera entendible, facilitando el uso de este.
		
	\item La tarjeta de prototipado ESP32 como unidad central de procesamiento dentro del desarrollo del hardware, representa una opción viable en la implementación de un sistema IoT, no solo por su bajo costo sino también porque sus características y funcionalidades están diseñadas para este tipo de aplicaciones, permitiendo así que el diseño y la implementación del sistema sea escalable.
	
	\item El uso del framework ESP-IDF, que se compone de un sistema operativo en tiempo real freeRTOS es muy util para realizar las diferentes funciones del sistema que comunica hardware con software, además de realizar las diferentes tareas y gestionar bien los recursos presentes en el chip ESP32, por lo tanto también se pueden agregar una gama mas amplia de funciones adicionales.
	
\end{itemize}