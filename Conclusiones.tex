\chapter{Conclusiones}

\begin{itemize}
	\item Hardware
	\item Durante el desarrollo del firmware con el uso del framework ESP-IDF se genera de manera adecuada las diferentes funcionalidades de la solución en cuestión, debido a que este es el encargado de comunicar el hardware con el software que se encuentra en la nube, gracias al uso de un sistema operativo en tiempo real se evidencia que la mayoria de tareas se realizan de manera adecuada y con respecto a este se puede seguir ampliando la gama de funciones a soportar por la tarjeta. 
	\item Realizar una aplicación web que se encargue de la gestión de los diferentes datos producidos por el hardware y que le brinde a este los datos para su control es una de las caracteristicas principales de las soluciones del internet de las cosas, ya que se tiene acceso a esta desde cualquier lugar con conexión a internet, además de comprender el funcionamiento de las PaaS y como usarlas de una manera adecuada.
	\item Al enfrentar el sistema con los usuarios finales este se ha desarrollado de la mejor forma posible, presentando los datos de manera entendible además de facilitar el uso de este.
	\item Sistema en General	
\end{itemize}