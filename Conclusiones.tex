\chapter{Conclusiones}

\begin{itemize}
	\item El sistema compuesto de hardware y software descrito en este documento es una solución IoT funcional, ya que este cuenta con la capacidad de monitorear y controlar un entorno de aplicación por medio de internet, el cual es en este caso una habitación de Smart House, siendo esta la representación de uno de los infinitos escenarios posibles a los cuales puede ser conectado el sistema, puesto que está diseñado con el fin de abarcar un amplio número de tareas, además de que acepta múltiples tipos de dispositivos de medida; combinando estos aspectos con una PaaS que permite la interacción humano-maquina, de tal manera que cumple con las características principales para una solución IoT.\\
	
	\item El hardware implementado para la solución IoT, funciona en conjunto con la aplicación del sistema embebido que enlaza su parte física con el software presente en la nube; este fue diseñado en dos etapas con un circuito para cada una, las cuales son la etapa de potencia AC y la etapa DC. Esta organización del sistema no solo permite una clasificación en cuanto a su funcionamiento, sino que también evita que los altos voltajes de la etapa AC causen algún tipo de interferencia de manera directa con la etapa DC. Cada etapa del hardware cuenta con una alimentación que garantiza los niveles correctos para la alimentación de los sensores y cargas, además de una apropiada lectura de cada dispositivo de medida y una adecuada manipulación del entorno.\\
	
	\item La aplicación web presentada en este trabajo, permite la interacción del usuario con los diferentes dispositivos que se encuentran conectados al hardware. Esta característica del sistema se encarga de la gestión de la información recolectada por el hardware junto con las órdenes administradas por el mismo, ya sea por visualización o manipulación del entorno de instalación, es decir, para su monitoreo o control, ya que se tiene acceso desde cualquier lugar con conexión a internet; de este modo la aplicación cumple con los diferentes alcances y puede seguir siendo ampliada a fin de adicionar funcionalidades al sistema. Con la implementación de este programa se han adquirido varios conocimientos sobre aplicaciones web y los lenguajes de programación que requieren, tales como PHP, JavaScript, entre otros.\\
	
	\item La prueba Beta realizada para evaluar el sistema con la participación de los usuarios finales otorgo una calificación de 4.7, lo cual permite confirmar que se ha desarrollado de la mejor forma posible, desde el encendido y manipulación del hardware junto con sus conexiones, hasta la navegación a través de la aplicación web, facilitando estos procesos con ayuda de un manual de uso donde se encuentran indicados y detallados, ya que este fue escrito con la intención de ser entendido y seguido por cualquier usuario. Con esta puntuación se demuestra que aún quedan algunos pequeños aspectos por enriquecer, pero también indica que cumple con la funcionalidad propuesta, con el objetivo de mejorar se debe mantener un estrecho contacto con el cliente recibiendo constantes opiniones.\\
	
	\item La tarjeta de prototipado ESP32 como unidad central de procesamiento dentro del desarrollo del hardware, representa una opción viable en la puesta en funcionamiento de un sistema IoT, no solo por su bajo costo sino también por sus características y funcionalidades, las cuales están diseñadas para este tipo de aplicaciones, permitiendo que el diseño y la implementación del sistema sea escalable, asimismo porque cuenta con un gran número de pines y diversas opciones para aumentar su capacidad.\\
	
	\item El uso del framework ESP-IDF, que se compone de un sistema operativo en tiempo real freeRTOS, es muy útil para realizar las diferentes funciones del sistema que comunica hardware con software, además de que permite ejecutar las diversas tareas y gestionar bien los recursos presentes en el chip ESP32, por lo tanto es posible agregar una amplia gama de características adicionales a la funcionalidad del sistema, gracias a que esta en constante expansión agregando nuevas opciones y diversos controladores para el ESP32.\\
	
	\item Se desarrolló una etapa de potencia AC que permite controlar las diferentes cargas presentes en una habitación, para esta se tiene un control de potencia por ángulo de fase el cual requiere de sincronización, precisión y un buen aislamiento del circuito con su unidad de control en este caso el ESP32. Por tal motivo se elige dicha tarjeta, ya que al poseer un RTOS es posible tener las características que brindan un correcto funcionamiento. Al implementar estos circuitos se han reforzado los diversos conocimientos obtenidos sobre los triacs y los distintos circuitos de aislamiento y activación para su uso.\\
	
\end{itemize}