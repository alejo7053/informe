\chapter{Conclusiones}

\begin{itemize}
	\item La tarjeta de prototipado ESP32 como unidad central de procesamiento dentro del desarrollo del hardware, representa una opción viable en la implementación de un sistema IoT, no solo por su bajo costo sino también porque sus características y funcionalidades están diseñadas para este tipo de aplicaciones, permitiendo así que el diseño y la implementación del sistema sea escalable.
	\item Durante el desarrollo del firmware con el uso del framework ESP-IDF, se generan de manera adecuada las diferentes funcionalidades de la solución en cuestión, debido a que este es el encargado de realizar una comunicación entre el hardware y el software que se encuentra en la nube, valiéndose de un sistema operativo en tiempo real para gestionar de manera adecuada dicha esta comunicación, gracias a esto se hace posible una ampliación en cuanto a la gama de funciones a soportar por la tarjeta. 
	\item La aplicación web se encarga de la gestión de la información producida por el hardware, ya sea por visualización o interacción del entorno de aplicación, es decir para su monitoreo o control, lo cual representan unas de características principales de las soluciones del internet de las cosas, ya que se tiene acceso a esta desde cualquier lugar con conexión a internet, además de comprender el funcionamiento de las PaaS y cómo usarlas de una manera adecuada.
	\item Al enfrentar el sistema con los usuarios finales este se ha desarrollado de la mejor forma posible, presentando los datos de manera entendible además de facilitar el uso de este.
	
\end{itemize}