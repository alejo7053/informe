\chapter{Datos Generales}

\section{Proponente(s), Director y Asesor(es)}
	PROPONENTES (s) \\ \\
	\fbox{\parbox{1\linewidth}{\textbf{Código Est:} \autorUnoCodigo   \hfill   \textbf{ Nombre:} \autorUnoTesis \\ 
			\textbf{Dirección:}  \autorUnoDireccion \hfill   \textbf{Teléfono:}\autorUnoTelefono \\ \\
			\textbf{E-mail:} \hspace{8cm} \textbf{Firma:} \hrulefill     \\
			\autorUnoEmail }} \\ \\ \\
	\fbox{\parbox{1\linewidth}{\textbf{Código Est:} \autorDosCodigo   \hfill   \textbf{ Nombre:} \autorDosTesis \\ 
			\textbf{Dirección:}  \autorDosDireccion \hfill   \textbf{Teléfono:}\autorDosTelefono \\ \\
			\textbf{E-mail:} \hspace{8cm} \textbf{Firma:} \hrulefill     \\
			\autorDosEmail }} \\ \\ \\
	\fbox{\parbox{1\linewidth}{\textbf{DIRECTOR} \hfill \textbf{Nombre:} \directorTesis \\ 
				\textbf{Títulos Universitario:} \directorTesisTitulo \\ 
				\textbf{Tiene Vinculación con la Universidad:} Si  \\ 
				\textbf{Teléfono:}\directorTesisTelefono \\ \\ 
				\textbf{E-mail:} \hspace{8cm} \textbf{Firma:} \hrulefill    \\ 
				\directorTesisEmail}} \\ \\ \\ 
	
	\section{Área}
	Este proyecto esta relacionado con las áreas de electrónica e Internet de las Cosas.
	\section{Modalidad}
	Trabajo de Grado
	\section{Titulo}
	Tarjeta Smart House para una habitación
	\section{Tema}
	Smart House es uno de los campos de aplicación del Internet de las Cosas (IoT) \cite{IdT}, el cual se enfoca principalmente en la interacción maquina a máquina (M2M) y maquina a persona (M2P) en una casa, el propósito de este concepto es la interconexión de objetos cotidianos que comúnmente no cuentan con acceso a internet o algún tipo de conexión. Todo esto con el fin de generar información de los objetos (por medio de sensores), tomando decisiones basadas en esta información (con una unidad central de procesamiento, como por ejemplo un microcontrolador) y finalmente, en algunas ocasiones, realizar una acción en base a la decisión tomada (por medio de actuadores), así como también darles uso a los datos, como, por ejemplo, para procesos estadísticos.  
	
	El tema será desarrollado durante la ejecución del proyecto, de acuerdo con la siguiente distribución:
	\begin{itemize}
		\item Búsqueda bibliográfica:			20\%
		\item Estudio y desarrollo teóricos:		30\%
		\item Desarrollo experimental en laboratorio:	40\%
		\item Documentación del Trabajo:		10\%
	\end{itemize}
	Las áreas de influencia del proyecto propuesto son esencialmente las comunicaciones digitales y control de potencia. \\
	
	\section{Palabras Clave}
	
	AC, Aplicación WEB, Circuitos Electrónicos, DC,Internet de las Cosas, Internet del Todo, Smart House.
	
	\section{Herramientas de busqueda}
	
	Google, Google Academico (scholar.google.com), Base de Datos Universidad del Quindío (www.uniquindio.edu.co/biblioteca): IEEE, ScienceDirect, Scopus