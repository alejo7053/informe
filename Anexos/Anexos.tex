\begin{appendix}
\chapter{Anexo: Tabla de Funciones en el Firmware}\label{AnexoA}



\chapter{Anexo: Encuesta de la Prueba Beta}\label{AnexoB}
\markboth{Anexo: Encuesta de la Prueba Beta}{Encuesta de la Prueba Beta}

\textbf{Evaluación de la Aplicación Smart House\\}

Califique del 1 al 5, siendo 5 la calificación máxima y 1 la mínima.\\

\begin{enumerate}
	\item El método para conectar la tarjeta a Internet es amigable e intuitivo.
		
\begin{tabbing}
	\hspace{2cm}\=\hspace{2cm}\=\hspace{2cm}\=\hspace{2cm}\=\kill
	(1)\>(2)  \>(3)  \>(4)  \>(5) 
\end{tabbing} 
		
	\item La interfaz para seleccionar la red e ingresar las credenciales es fácil de utilizar. 
	
	\begin{tabbing}
		\hspace{2cm}\=\hspace{2cm}\=\hspace{2cm}\=\hspace{2cm}\=\kill
		(1)\>(2)  \>(3)  \>(4)  \>(5) 
	\end{tabbing} 

	\item Es fácil volver a configurar la conexión a Wi-Fi después de iniciado el funcionamiento.
	
	\begin{tabbing}
		\hspace{2cm}\=\hspace{2cm}\=\hspace{2cm}\=\hspace{2cm}\=\kill
		(1)\>(2)  \>(3)  \>(4)  \>(5) 
	\end{tabbing} 

	\item El inicio de sesión en la pagina web es claro.
	
\begin{tabbing}
	\hspace{2cm}\=\hspace{2cm}\=\hspace{2cm}\=\hspace{2cm}\=\kill
	(1)\>(2)  \>(3)  \>(4)  \>(5) 
\end{tabbing} 

	\item Los datos de los sensores y estados de las salidas se presentan de una manera clara y entendible.

\begin{tabbing}
	\hspace{2cm}\=\hspace{2cm}\=\hspace{2cm}\=\hspace{2cm}\=\kill
	(1)\>(2)  \>(3)  \>(4)  \>(5) 
\end{tabbing} 
	
	\item Encender o apagar un dispositivo o salida es amigable para el usuario y se presenta de forma clara.
	
\begin{tabbing}
	\hspace{2cm}\=\hspace{2cm}\=\hspace{2cm}\=\hspace{2cm}\=\kill
	(1)\>(2)  \>(3)  \>(4)  \>(5) 
\end{tabbing} 

	\item El tiempo de respuesta después de encender o apagar un dispositivo es bueno.

	\begin{tabbing}
		\hspace{2cm}\=\hspace{2cm}\=\hspace{2cm}\=\hspace{2cm}\=\kill
		(1)\>(2)  \>(3)  \>(4)  \>(5) 
	\end{tabbing} 

	\item Las reglas para los dispositivos o salidas son fáciles de configurar y eliminar.

\begin{tabbing}
	\hspace{2cm}\=\hspace{2cm}\=\hspace{2cm}\=\hspace{2cm}\=\kill
	(1)\>(2)  \>(3)  \>(4)  \>(5) 
\end{tabbing} 

	\item La aplicación web es fácil de navegar.

\begin{tabbing}
	\hspace{2cm}\=\hspace{2cm}\=\hspace{2cm}\=\hspace{2cm}\=\kill
	(1)\>(2)  \>(3)  \>(4)  \>(5) 
\end{tabbing} 

	\item La aplicación web tiene una interfaz fácil de usar.

\begin{tabbing}
	\hspace{2cm}\=\hspace{2cm}\=\hspace{2cm}\=\hspace{2cm}\=\kill
	(1)\>(2)  \>(3)  \>(4)  \>(5) 
\end{tabbing} 

	\item La aplicación web tiene una combinación de colores agradable.

\begin{tabbing}
	\hspace{2cm}\=\hspace{2cm}\=\hspace{2cm}\=\hspace{2cm}\=\kill
	(1)\>(2)  \>(3)  \>(4)  \>(5) 
\end{tabbing} 

\end{enumerate}

\chapter{Anexo: Nombrar el anexo C de acuerdo con su contenido}
MANEJO DE LA BIBLIOGRAF\'{I}A: la bibliograf\'{\i}a es la relaci\'{o}n de las fuentes documentales consultadas por el investigador para sustentar sus trabajos. Su inclusi\'{o}n es obligatoria en todo trabajo de investigaci\'{o}n. Cada referencia bibliogr\'{a}fica se inicia contra el margen izquierdo.\\

La NTC 5613 establece los requisitos para la presentaci\'{o}n de referencias bibliogr\'{a}ficas citas y notas de pie de p\'{a}gina. Sin embargo, se tiene la libertad de usar cualquier norma bibliogr\'{a}fica de acuerdo con lo acostumbrado por cada disciplina del conocimiento. En esta medida es necesario que la norma seleccionada se aplique con rigurosidad.\\

Es necesario tener en cuenta que la norma ISO 690:1987 (en Espa\~{n}a, UNE 50-104-94) es el marco internacional que da las pautas m\'{\i}nimas para las citas bibliogr\'{a}ficas de documentos impresos y publicados. A continuaci\'{o}n se lista algunas instituciones que brindan par\'{a}metros para el manejo de las referencias bibliogr\'{a}ficas:\\

\begin{center}
\centering%
\begin{tabular}{|p {7.5 cm}|p {7.5 cm}|}\hline
\arr{Instituci\'{o}n}&Disciplina de aplicaci\'{o}n\\\hline%
Modern Language Association (MLA)&Literatura, artes y humanidades\\\hline%
American Psychological Association (APA)&Ambito de la salud (psicolog\'{\i}a, medicina) y en general en todas las ciencias sociales\\\hline
Universidad de Chicago/Turabian &Periodismo, historia y humanidades.\\\hline
AMA (Asociaci\'{o}n M\'{e}dica de los Estados Unidos)&Ambito de la salud (psicolog\'{\i}a, medicina)\\\hline
Vancouver &Todas las disciplinas\\\hline
Council of Science Editors (CSE)&En la actualidad abarca diversas ciencias\\\hline
National Library of Medicine (NLM) (Biblioteca Nacional de Medicina)&En el \'{a}mbito m\'{e}dico y, por extensi\'{o}n, en ciencias.\\\hline
Harvard System of Referencing Guide &Todas las disciplinas\\\hline
JabRef y KBibTeX &Todas las disciplinas\\\hline
\end{tabular}
\end{center}

Para incluir las referencias dentro del texto y realizar lista de la bibliograf\'{\i}a en la respectiva secci\'{o}n, puede utilizar las herramientas que Latex suministra o, revisar el instructivo desarrollado por el Sistema de Bibliotecas de la Universidad Nacional de Colombia\footnote{Ver: www.sinab.unal.edu.co}, disponible en la secci\'{o}n "Servicios", opci\'{o}n "Tr\'{a}mites" y enlace "Entrega de tesis".

\end{appendix}
