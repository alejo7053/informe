\begin{appendix}
\chapter{Anexo: Tabla de Funciones en el Firmware}\label{AnexoA}

\begin{table}[H]
	\begin{center}
		\caption{Lista de Interrupciones programadas.}
		\label{table:inte}
		\begin{tabular}{|c|p{5cm}|}
			\hline 
			Nombre & Descripción \\ 
			\hline 
			detector & Señal de detección de cruce por cero para sincronizar con la red eléctrica y el control de las cargas AC \\ 
			\hline 
			gpio\_wifi\_rst & Para borrar las credenciales wifi, habilitando un semáforo\\ 
			\hline 
			gpio\_states\_sensors & Leer los sensores de estado, habilitando una cola, con la información del pin que la genera. \\ 
			\hline 
		\end{tabular} 
	\end{center}
\end{table}
	
\begin{table}[H]
	\begin{center}
		\caption{Lista de Tareas programadas.}
		\label{table:tasks}
		\begin{tabular}{|c|c|c|p{5cm}|}
			\hline 
			Nombre & Tiempo Suspendida [s] & Prioridad & Descripción \\ 
			\hline 
			wifi\_manager & 0 & 4 & Esta tarea se encarga de gestionar las conexiones wi-fi. \\ 
			\hline 
			http\_server & 0.1 & 5 & Encargada de servir una interfaz para seleccionar la red wi-fi a conectar la tarjeta \\ 
			\hline 
			http\_get\_task & 1 & 5 & Realiza la petición a la aplicación web, enviando y recibiendo datos de esta. \\ 
			\hline 
			air\_sensor\_read& 60 & 3 & Lectura del sensor de calidad de aire. \\ 
			\hline 
			simple\_loads\_task& 1 & 4 & Controla las cargas AC-On/Off  y las DC sin inversión de giro. \\ 
			\hline 
			PWM\_task& 1 & 5 & Controla las cargas con inversión de giro. \\ 
			\hline 
			ControlLoad\_task& 1.5 & 7 & Controla las cargas AC por ángulo de fase. \\ 
			\hline 
			i2c\_task& 1 & 2 & Lectura de los dispositivos I2C. \\ 
			\hline 
			gpio\_task& - & 7 & Lectura de los dispositivos de estado por medio de una interrupción qué se comunica con esta a través de una cola. \\ 
			\hline 
			dht\_task& 60 & 3 & Lectura del sensor de temperatura y humedad. \\ 
			\hline 
			ResetWifi& - & 5 & Por medio de una interrupción que librea un semáforo, elimina las credenciales de wifi ingresadas. \\ 
			\hline 
			Audio\_task& - & 3 & Sincronizado con un grupo de eventos, se activa cuanto se utilizan las reglas. \\ 
			\hline 
		\end{tabular} 
	\end{center}
\end{table}

\chapter{Anexo: Encuesta de la Prueba Beta}\label{AnexoB}
\markboth{Anexo: Encuesta de la Prueba Beta}{Encuesta de la Prueba Beta}

\textbf{Evaluación de la Aplicación Smart House\\}

Califique del 1 al 5, siendo 5 la calificación máxima y 1 la mínima.\\

\begin{enumerate}
	\item El método para conectar la tarjeta a Internet es amigable e intuitivo.
		
\begin{tabbing}
	\hspace{2cm}\=\hspace{2cm}\=\hspace{2cm}\=\hspace{2cm}\=\kill
	(1)\>(2)  \>(3)  \>(4)  \>(5) 
\end{tabbing} 
		
	\item La interfaz para seleccionar la red e ingresar las credenciales es fácil de utilizar. 
	
	\begin{tabbing}
		\hspace{2cm}\=\hspace{2cm}\=\hspace{2cm}\=\hspace{2cm}\=\kill
		(1)\>(2)  \>(3)  \>(4)  \>(5) 
	\end{tabbing} 

	\item Es fácil volver a configurar la conexión a Wi-Fi después de iniciado el funcionamiento.
	
	\begin{tabbing}
		\hspace{2cm}\=\hspace{2cm}\=\hspace{2cm}\=\hspace{2cm}\=\kill
		(1)\>(2)  \>(3)  \>(4)  \>(5) 
	\end{tabbing} 

	\item El inicio de sesión en la pagina web es claro.
	
\begin{tabbing}
	\hspace{2cm}\=\hspace{2cm}\=\hspace{2cm}\=\hspace{2cm}\=\kill
	(1)\>(2)  \>(3)  \>(4)  \>(5) 
\end{tabbing} 

	\item En la aplicación web, los datos de los sensores y estados de las salidas se presentan de una manera clara y entendible.

\begin{tabbing}
	\hspace{2cm}\=\hspace{2cm}\=\hspace{2cm}\=\hspace{2cm}\=\kill
	(1)\>(2)  \>(3)  \>(4)  \>(5) 
\end{tabbing} 
	
	\item Encender o apagar un dispositivo o salida es amigable para el usuario y se presenta de forma clara.
	
\begin{tabbing}
	\hspace{2cm}\=\hspace{2cm}\=\hspace{2cm}\=\hspace{2cm}\=\kill
	(1)\>(2)  \>(3)  \>(4)  \>(5) 
\end{tabbing} 

	\item El tiempo de respuesta después de encender o apagar un dispositivo es bueno.

	\begin{tabbing}
		\hspace{2cm}\=\hspace{2cm}\=\hspace{2cm}\=\hspace{2cm}\=\kill
		(1)\>(2)  \>(3)  \>(4)  \>(5) 
	\end{tabbing} 

	\item Las reglas para los dispositivos o salidas son fáciles de configurar y eliminar.

\begin{tabbing}
	\hspace{2cm}\=\hspace{2cm}\=\hspace{2cm}\=\hspace{2cm}\=\kill
	(1)\>(2)  \>(3)  \>(4)  \>(5) 
\end{tabbing} 

	\item La aplicación web es fácil de navegar.

\begin{tabbing}
	\hspace{2cm}\=\hspace{2cm}\=\hspace{2cm}\=\hspace{2cm}\=\kill
	(1)\>(2)  \>(3)  \>(4)  \>(5) 
\end{tabbing} 

	\item La aplicación web tiene una interfaz fácil de usar.

\begin{tabbing}
	\hspace{2cm}\=\hspace{2cm}\=\hspace{2cm}\=\hspace{2cm}\=\kill
	(1)\>(2)  \>(3)  \>(4)  \>(5) 
\end{tabbing} 

	\item La aplicación web tiene una combinación de colores agradable.

\begin{tabbing}
	\hspace{2cm}\=\hspace{2cm}\=\hspace{2cm}\=\hspace{2cm}\=\kill
	(1)\>(2)  \>(3)  \>(4)  \>(5) 
\end{tabbing} 

\end{enumerate}

\chapter{Anexo: CD}

Se anexa un CD con los códigos de la aplicación web y el firmware, una versión digital de este documento y un manual de usuario explicando el uso de toda la solución IoT.\\

Además el enlace a la página principal del servidor: http://smarthouseuq.herokuapp.com/

\end{appendix}
