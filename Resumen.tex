\newpage
\textbf{\LARGE Resumen}
\addcontentsline{toc}{chapter}{\numberline{}Resumen}\\\\
Este documento detalla el desarrollo y la implementación de una sistema de monitoreo y control para una habitación en un entorno de Smart House basado en el chip ESP32 que esta diseñado para el uso en aplicaciones que usan el paradigma de Internet de las cosas (IoT). La tarjeta permite la conexión de diferentes sensores y dispositivos, por medio de el diseño de diferentes circuitos para su adecuación y correcto funcionamiento; para este caso se utilizan los sensores más comunes en dicho entorno, como lo son, sensor de movimiento, temperatura, luminosidad, entre otros. Para la parte de control se tienen salidas de AC y DC con la posibilidad de variar la energía entregada a esta o simplemente encender y apagar. En cuanto a la visualización de los datos obtenidos de los sensores y también la interacción del usuario con los diferentes elementos conectados a la tarjeta se desarrolla una aplicación web implementada en Laravel, framework de desarrollo de aplicaciones web que utiliza PHP, esta aplicación web funciona a través de Heroku, una plataforma como servicio (PaaS), que proporciona toda la infraestructura necesaria para su funcionamiento.\\
\textbf{\small Palabras clave: Aplicación Web, ESP32, Internet de las Cosas, Monitoreo,Smart House}.\\[2.0cm]
\textbf{\LARGE Abstract}\\\\
This document .\\[2.0cm]
\textbf{\small Keywords: palabras clave en ingl\'{e}s(m\'{a}ximo 10 palabras, preferiblemente seleccionadas de las listas internacionales que permitan el indizado cruzado)}\\