\newpage
\thispagestyle{empty} \textbf{}\normalsize
\textbf{\LARGE Resumen}
\markboth{}{}
\addcontentsline{toc}{chapter}{\numberline{}Resumen}\\\\
Este documento detalla el desarrollo y la implementación de una sistema de monitoreo y control para una habitación en un entorno de Smart House basado en el chip ESP32, el cual esta diseñado para su implementacion en aplicaciones que usan el paradigma de Internet de las cosas (IoT). La tarjeta permite la conexión de diferentes sensores y dispositivos, por medio del diseño de diferentes circuitos para su adecuación y correcto funcionamiento; para este caso se utilizan sensores que midan magnitudes y características comunes en dicho entorno, como lo son, movimiento, temperatura, luminosidad, entre otros. Para la parte de control se tienen salidas de AC y DC con la posibilidad de variar la energía entregada a esta o simplemente encender y apagar. En cuanto a la visualización de los datos obtenidos de los sensores y también la interacción del usuario con los diferentes elementos conectados a la tarjeta, se desarrolla una aplicación web implementada en Laravel, framework de desarrollo de aplicaciones web que utiliza PHP, esta aplicación web funciona a través de Heroku, una plataforma como servicio (PaaS), que proporciona toda la infraestructura necesaria para su funcionamiento.\\
\textbf{\small Palabras clave: Aplicación Web, ESP32, Internet de las Cosas, Monitoreo,Smart House}.\\[1.0cm]
\textbf{\LARGE Abstract}\\\\
This document details the development and implementation of a monitoring and control system for a room in a Smart House environment based on the ESP32 chip, which is designed for implementation in applications that use the Internet of Things paradigm (IoT ). The card enables the connection of different sensors and devices through different circuits design for adequacy and proper functioning; for this case, sensors are used that measure magnitudes and common characteristics in said environment, such as movement, temperature, luminosity, among others. For the control part there are AC and DC outputs with the possibility of varying the energy delivered to it or simply turning it on and off. Regarding the visualization of the data obtained from the sensors and also the user interaction with the different elements connected to the card, a web application implemented in Laravel is developed, a web application development framework that uses PHP, this web application works through Heroku, a platform as a service (PaaS), which provides all the necessary infrastructure for its operation.\\[1.0cm]
\textbf{\small Keywords: ESP32, Internet of Things, Monitoring, Smart House, Web Application}