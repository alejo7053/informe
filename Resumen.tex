\newpage
\thispagestyle{empty} \textbf{}\normalsize
\textbf{\LARGE Resumen}
\markboth{}{}
\addcontentsline{toc}{chapter}{\numberline{}Resumen}\\\\
 Este documento detalla el desarrollo e implementación de un sistema de monitoreo y control para una habitación en un entorno de Smart House basado en el chip ESP32, el cual está diseñado con un enfoque hacia aplicaciones que usan el paradigma de Internet de las cosas (IoT). La tarjeta permite la conexión de diferentes dispositivos, por medio del diseño de diversos circuitos para su adecuación y correcto funcionamiento; en el presente caso, se utilizan sensores que miden magnitudes y características comunes en dicho ambiente, como lo son, movimiento, temperatura, luminosidad, entre otros. La etapa de control tiene salidas de Corriente Alterna (AC) y Corriente Directa (DC), con la posibilidad de variar la energía entregada a esta o simplemente encender y apagar. En cuanto a la visualización de los datos obtenidos de los sensores y la interacción del usuario con los distintos elementos conectados a la tarjeta, se desarrolla una aplicación web implementada en Laravel, framework de desarrollo de aplicaciones web que utiliza PHP, que funciona a través de Heroku, una Plataforma como Servicio (PaaS), que proporciona toda la infraestructura necesaria para su operación. El desempeño del sistema es comprobado con usuarios finales a través de los diferentes ítems de una prueba Beta propuesta.\\
 
\textbf{\small Palabras clave: ESP32, Internet de las Cosas, Smart House}.\\[1.0cm]

\textbf{\LARGE Abstract}\\\\
This document details the development and implementation of a monitoring and control system for a room in a Smart House environment based on the ESP32 chip, which is designed with a focus on applications that use the Internet of Things paradigm (IoT) . The board allows the connection of different devices by means of the design of various circuits for its adaptation and correct operation; in the present case, sensors are used to measure variables and common features in this environment, such as motion, temperature, luminosity, among others. The control stage has outputs for Alternating Current (AC) and Direct Current (DC), with the possibility of varying the delivered energy, or simply turning it on and off. For visualization of the data obtained from the sensors and user interaction with the different elements connected to the board, a web application is implemented in Laravel. This web application was development framework in PHP, which works through Heroku, a Platform as a Service (PaaS), which provides all the necessary infrastructure for its operation. The performance of the system is verified with end users through different items proposed in a beta test.\\

\textbf{\small Keywords: ESP32, Internet of Things, Smart House}