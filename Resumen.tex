\newpage
\thispagestyle{empty} \textbf{}\normalsize
\textbf{\LARGE Resumen}
\markboth{}{}
\addcontentsline{toc}{chapter}{\numberline{}Resumen}\\\\
 Este documento detalla el desarrollo y la implementación de una sistema de monitoreo y control para una habitación en un entorno de Smart House basado en el chip ESP32, el cual está diseñado con un enfoque hacia aplicaciones que usan el paradigma de Internet de las cosas (IoT). La tarjeta permite la conexión de diferentes sensores y dispositivos, por medio del diseño de múltiples etapas de un circuito para su adecuación y correcto funcionamiento; para este caso, se utilizan sensores que midan magnitudes y características comunes en dicho entorno, como lo son, movimiento, temperatura, luminosidad, entre otros. Para la etapa de control se tienen salidas de corriente alterna (AC) y corriente directa (DC), con la posibilidad de variar la energía entregada a esta o simplemente encender y apagar. En cuanto a la visualización de los datos obtenidos de los sensores y la interacción del usuario con los diferentes elementos conectados a la tarjeta, se desarrolla una aplicación web implementada en Laravel, framework de desarrollo de aplicaciones web que utiliza PHP, la cual funciona a través de Heroku, una plataforma como servicio (PaaS), que proporciona toda la infraestructura necesaria para su funcionamiento.\\
\textbf{\small Palabras clave: ESP32, Internet de las Cosas, Smart House}.\\[1.0cm]
\textbf{\LARGE Abstract}\\\\
This document details the development and implementation of a monitoring and control system for a room in a Smart House environment based on the ESP32 chip, which is designed with a focus on applications that use the Internet of Things paradigm (IoT ). The card allows the connection of different sensors and devices, through the design of multiple stages of a circuit for its adequacy and correct operation; for this case, sensors are used that measure magnitudes and common characteristics in said environment, such as, movement, temperature, luminosity, among others. For the control stage there are alternating current (AC) and direct current (DC) outputs, with the possibility of varying the energy delivered to it or simply turning it on and off. As for the visualization of the data obtained from the sensors and the user's interaction with the different elements connected to the card, a web application implemented in Laravel is developed, a web application development framework that uses PHP, which works through of Heroku, a platform as a service (PaaS), which provides all the necessary infrastructure for its operation.\\[1.0cm]
\textbf{\small Keywords: ESP32, Internet of Things, Smart House}